
\documentclass[preprint,authoryear]{elsarticle}

\usepackage[utf8]{inputenc}
\usepackage{xcolor}
\usepackage{amsmath}

\newcommand{\memo}[2]{\textcolor{#1}{#2}}
\newcommand{\xavi}[1]{\memo{orange}{xavi: #1\\}}
\newcommand{\enrico}[1]{\memo{blue}{enrico: #1\\}}

\journal{Theoretical Population Biology}

\begin{document}

\begin{frontmatter}

\title{Allee Effect and Cultural Evolution}
%\title{The evolution of payoff-biased learning under Allee Effect} %Perhaps a 


\author[label1,label2]{Enrico R. Crema\corref{cor1}}
\cortext[cor1]{Corresponding Author: enrico.crema@upf.edu}
\author[label3]{Xavier Rubio-Campillo}

\address[label1]{CaSEs - Complexity and Socio-Ecological Dynamics Research Group, Barcelona}
\address[label2]{UCL Institute of Archaeology}
\address[label3]{BSC - Barcelona Supercomputing Center}



\begin{abstract}
Several forms of social learning rules are guided by the amount of payoff received by social teachers adopting specific cultural variants. The rationale, is that such payoff will be correlated to the beneficial or detrimental properties of a given variant. Here we assume that the payoff directly affects the reproductive fitness of individuals, and that it is determined by the density of individuals possessing the same cultural variant. More specifically we consider a  form of density-dependence where a too small or too large population of individuals possessing the same variant will be subject to a decline in fitness, while an intermediate size is optimal. This non-linear relationship, known as Allee effect in ecology, portrays a large number of cultural variants, in particular subsistence strategies that are enhanced  by the effect of cooperation but are constrained by interference and limited resources. This paper explores the evolutionary dynamics of a two trait model where each variant exhibit Allee effect. We used a modified Lotka-Volterra model of payoff-biased social learning, and show that that different degrees of reliance on social learning can exhibit a variety of equilibria, including episodes of reversion to suboptimal variants, long-term co-existence of multiple variants, and fixation to a single optimal or suboptimal strategy.
\end{abstract}

\begin{keyword}
Social Learning\sep Subsistence\sep Allee Effect\sep Cooperation\sep Carrying Capacity
\end{keyword}

\end{frontmatter}

\section{Introduction}

A key element in all social learning strategies is the availability of some heuristics and learning bias that guides the selection of a given variant over another. This information can be based on some properties of the variant itself (\emph{content-bias}) or derived from contextual cues (\emph{context-bias}), such as the rarity or the commonness or the rareness of a trait, or some features associated with the model (i.e. the social teacher) such as prestige, success, or the similarity to the learner \citep{henrich_mcelreath2003}. An expected consequence of \emph{context-bias} is the possibility of a feedback process, whereby the attractiveness of a given might change as a consequence of the learning strategy \citep{kendal_etal_2009}. For example a strong anti-conformist bias might initially favour rare traits, but as they spread into a population their selective advantage will decline as the result of a negative feedback. In a similar, but in a opposite direction, both conformist-bias and homophily-guided learning strategies would exercise a positive feedback loop.  %citation needed?

This paper examines a group of learning strategies generally referred to as \emph{payoff-based transmission} \citep{schlag1998,kendal_etal_2009,lake_and_crema_2012,baldini2013,kandler_and_laland_2013,crema_lake_inpress}. The core assumption shared by these models is that the probability of copying a given cultural variant is positively correlated with some payoff signal expressed by individuals possessing such a variant. Several variants of this learning strategies have been proposed and explored, including: "imitate the best", where the social learner identifies and copies the trait possessed by the individual with the highest payoff; "compare means" where the social learner compare the average payoff of different strategies and chooses the highest variant; and the "proportional imitation rule", where the probability of adopting a variant is proportional to the difference in payoff between the demonstrator and the learner \citep{schlag1998,baldini2013}. 

Whilst most studies focused on how the difference in these imitation rules can drive the dynamics of the system, less attention has been paid on the payoff signal.  The exact nature of the signal is most likely some indirect proxy that is assumed, by the social learner, to be linked with the target cultural trait. Thus one might be guided by a variety of cues (e.g. number of offpsring, income, presence/absence of other traits associated with success or prestige, etc.) that are differently correlated with the target trait resulting into different degrees of uncertainty in the signal. The magnitude of this uncertainty can have strong impact to efficiency of specific learning rules. For example, Crema and Lake (In press) have shown that "imitate the best" is negatively correlated with payoff uncertainty, When this is high novel traits, which are expected to be initially rare in a population, have a decreased probability of being selected even if their average payoff is higher than the extant suboptimal variants. 

Given the same cultural trait, the variability in the payoff signal is partly derived by the presence of interacting cultural and physical traits, as well as the environment where the trait is manifest. For example, a fisherman might evaluate the performance of a hook (the target cultural trait) using as a cue the number of fish captured (the payoff signal). The correlation between the two is partly due to the efficiency of the hook, but also by the rod, the choice of the bait, the availability of the prey species, the choice of the fishing spot, the physical properties of the fisherman, and chance. In many cases we should expect that the social teacher and learner share many of these interacting traits as well as the environment where these are displayed. In other words, cultural and ecological inheritance can generally reduce the uncertainty and the variability in the payoff-signal. There are however exceptions. One of these are contexts where the success of a variant is strongly correlated with the absolute number of individuals possessing the same trait within a population. This density dependence implies the adoption (or abandonment) of a trait can alter the expected payoff-signal, possibly generating a variety of feedback mechanism.  

Several cultural traits are expected to show such density dependence. For example, behaviour linked to the exploitation of limited resources are expected to exhibit a negative density dependence. Novel optimal variants are likely to spread rapidly when the trait is rare, but once a threshold frequency is exceeded,  beneficial effect will start to decline, the payoff will become smaller, and eventually rarer suboptimal variants can be preferred (see Lake and Crema 2012 for an extensive analysis). Density dependence can also have a positive direction, whereby an increase in the number of individuals possessing a trait might have a selective advantage additional to what is already expected from its frequency. Variants that are enhanced by some form of cooperation, or more in general traits exhibiting a positive niche construction are expected to show such a pattern, known in ecology as Allee Effect \citep{allee1958}. For example foraging groups are known to show higher performance when the number of individuals is not too small \citep{hill_and_hawkes_1983,janssen_and_hill_2014}, while certain activity such as anthropogenic fire \citep{bird2013} or selective hunting \citep{dods_2002} might require a population density beyond a certain threshold in order to promote sufficient niche enhancement and a consequent increase in payoff \citep{rowley-conwy_and_layton_2011}. Other traits that might be promoted by increased density includes language \citep{kandler2009}, organisational populations \citep{caroll_and_hannan_1989}, % and prosocial behaviour. 
%I think Ostrom 2000 was added here in relation to this. In any case we need one extra example here. 

Here we explore the combined effect of positive and negative density-dependent selection, assuming that: 1) the social learning strategy is payoff-based transmission; and 2)  payoff is equivalent to the reproductive fitness of each individual. We will demonstrate, by exploring a a two-trait difference equation, that the degree by which a population relies on social learning can strongly affect the evolutionary trajectory of the system, leading to the emergence of stable and unstable equilibria, as well as the occurrence of temporary or permanent episodes of reversion to suboptimal variants. 


\section{The Model}
We consider the evolutionary dynamics of two mutually exclusive subsistence strategies which fitness have both positive and negative density dependence. In particular we examine how two population of individuals, $a$ and $b$, change over time if we allow for a social learning strategy guided by a ``proportional imitation rule'', that is the probability of an individual with lower payoff to adopt a the trait of an individual with higher payoff is proportional to the difference in the payoff. The model can theoretically be extended to $n$ strategies, but here we decide to explore a two trait version for simplicity. The core model can be depicted with the following coupled difference equation:

\begin{align}
\label{eq1}
\begin{cases}
a_{t+1}& = a_t + a_t R_{a,t} + C_{a,b,t} \\
b_{t+1}& = b_t + b_t R_{b,t} + C_{a,b,t}
\end{cases}
\end{align}

where $a_{t+1}$ and $b_{t+1}$ are the population of each trait at time $t+1$. This is given by the number of individuals during the previous time-step ($a_t$ and $b_t$), the reproductive rate of each population ($R_{a,t}$ and $R_{b,t}$), and the shift from one population to another dictated by cultural transmission ($C_{a,b,t}$). 

The reproductive growth rate of each population is defined by the following paired equation:

\begin{align}
\label{eq2}
\begin{cases}
R_{a,t}& = r_a (\frac{a_t}{A_a}-1)(1-\frac{a_t}{K_a})\\
R_{b,t}& = r_b (\frac{b_t}{A_b}-1)(1-\frac{b_t}{K_b}) 
\end{cases}
\end{align}

where $r$ is the basic growth rate, $A$ is the \emph{Allee} threshold,  $K$ is the carrying capacity, and the condition $A < K$. Equation \eqref{eq2} implies that the reproductive rate is negative when density is below $A$ or above $K$, and will reach its maxima $R_{max}=\frac{r(A-K)2}{4AK}$ when the population density is equal to $\frac{(K-A)}{2}$. 

Cultural transmission is portrayed in \eqref{eq3}:

\begin{align}
\label{eq3}
C_{a,b,t} = 
\begin{cases}
0& \text{if } R_{a,t} = R_{b,t}\\
\zeta(a_t+a_tR_{a,t})& \text{if } R_{a,t} > R_{b,t}\\
-\zeta(b_t+b_tR_{b,t})& \text{if } R_{a,t} < R_{b,t}
\end{cases}
\end{align}

where $\zeta$ defines the proportion of individuals changing strategy, and given by \eqref{eq4}:

\begin{align}
\label{eq4}
\zeta = 
\begin{cases}
z& \text{if }|R_{a,t}-R_{b,t}| > b\Delta\\
z\frac{|R_{a,t}-R{b,t}|}{m\Delta}& \text{if }|R_{a,t}-R_{b,t}| \leq b\Delta\\
\end{cases}
\end{align}

where $z$ is a measure of reliance in social learning (effectively equivalent to highest possible transmission rate),  $\Delta$ is the maximum between $R_{max(a,t)}$ and $R_{max(b,t)}$, and $m$ is a calibration parameter that measure the perception of the difference in the payoffs (i.e. small values of $m$ determines a higher rate by which the theoretical maximum transmission rate is reached). Thus \eqref{eq3} and \eqref{eq4}  conforms with the ``proportional imitation rule''  \citep{schlag1998}, although here the payoff is equivalent to the reproduction rate. 

\section{Results}

We investigate the equilibrium conditions for different initial population sizes of  $a$ and $b$ and different settings of $z$, $A$, and $K$, assuming that the $r_a=r_b$, and for convenience holding the condition $K_b>K_a$. Given that equations \ref{eq1},~\ref{eq4} have no analytical solution, we identified equilibrium conditions via simulation (see appendix 1 for details; source codes can be found on the following repository %put link here
as well as in the ESM).  
%if the points we need to reiterate are short perhaps a footnote will suffice instead of an appendix? 



\subsection{Equilibrium conditions with no social learning, $z=0$}

\subsection{Effect of social learning $z>0$}

\subsection{Effect of competition}





\section{Experiments}

We investigate different initial conditions of $m$ and $n$ for 6 different values of z (0,0.05,0.1,0.2,0.3, and 0.5). For each parameter combination we apply equations (\ref{eq1},~\ref{eq4}) for 50,000 iterations and record the final values of $m$ and $n$, identifying the following equilibria.

\enrico{We might need to use a rounding factor. And in fact these introduces to the problem of scale and difference between discrete and continuos  models.}
\enrico{We can analytical solve equations 1~4 to shows that the E1, E2 and E3 are indeed equilibria. Perhaps in an appendix? (although this is quite obvious...)}

We are aware that condition U can potentially become one of the other stable equilibria after the 50,000 simulations, hence we computed the Lyapunov exponent to define the stability of the system. We devised three core scenarios reflecting fixing b=2 and equating Rn to gRm, where g is given by the following equation:

\xavi{equation5, lyapunov, where this does go?}

so that when Rn=gRm,  Rmax(m,t) and Rmax(n,t) becomes equal. In all scenarios we considered Allee thresholds and carrying capacities as proportions by ensuring that max(Km,Kn)=1.

\enrico{Justify choice of arbitrary values}

\subsection{Scenario 1: Traits with identical payoffs (i.e. Am = An   and Km=Kn)}

The first scenario assumes two traits that are mathematically equivalent, so that $A_m = A_m, K_m=K_n and r_m=r_n$. Figure 1 shows the basin of attractions of each of the four equilibria for different values of z, setting r=0.005, .A=0.2 and K=1. When the z=0, i.e. when cultural transmission is disabled we have straight boundaries between the basins of E1, E2m, E2n, and E3 defined by Allee threshold. In other words when both one or both the population are below A we have some extinction events (E3 when mt=0<Am and nt=0<An, E2m when mt=0>Am and nt=0<An and E3n when mt=0<Am and nt=0>An). When z>0, the boundaries of these basins changes and with high z we see the emergence of unstable equilibria (U). 


In all cases the E1 equilibria is maintained when $mt=0=nt=0 > Am=An$ as in this scenario differences arises only when m DIFFERENT OF n, and when the two populations are equally sized, C(m,n) is equal to 0 (see equation [3]).  

\subsection{Scenario 2: Overlapping payoffs (i.e.  Am <  An  < Km < Kn)}

Am=0.2, An=0.5, Km=0.7,Kn=1

\subsection{Scenario 3: Internal overlap  (i.e.  Am <  An < Kn <  Km )}

An=0.2, Am=0.5, Km=0.7,Kn=1

\subsection{A note on competition}

The model described in equations  (\ref{eq1},~\ref{eq4}) assumes that the resource pool of the two strategy are completely independent to each other. In reality many strategies have some degree of competition, whereby the frequency dependence is not exclusively internal. In other words, the frequency of one  trait might increase or decrease the fitness of the other. The following equation can substitute equation [2]: 

\xavi{equation6}

where the four new parameters cAn, cKn, cAm, and cKm, measure respectively the negative (or positive) effect determined by the number of individuals of the “other” traits. When cA>0 the Allee threshold is essentially increased, while when cK>0 the carrying capacity is diminished as function of the other population., with larger values of c dictates a stronger influence of the other population. 

One simple scenario is to assume that one strategy, say n, is dominant over its alternative m. We can  set cAn=0.1,  cKn=0.1,  cAm=-0.1, and  cKm=-0.1. Here negative values of c indicate a positive enhancement of one strategy over the other (in this case n gets benefit to the expense of m). 

\section{Discussion}

%General Intro on Human History%
Major transitions in human history have been marked by the fast spread of key innovations, which laid the foundations of further cumulative development in economy, society, and culture. Shifts in subsistence economy has often been regarded as a prime example of these transitions, triggered by the spread of a suite of behavioural traits such as the know-how of the domestication process \citep{barker2006}; the use of new subsistence tools %citation needed
and processing techniques \citep{molleson1993}, and the inclusion/exclusion and extensification/intensification of different prey species %citation needed
Archaeological evidence does  not support a simple progressive adoption of these key cultural variants, and instead a showcase a variety of trajectories, including the long-term coexistence of multiple subsistence strategies, or even reversions to older traits \citep{rowley2001}.  
  %add long list of citations. or perhaps one or two concrete examples (probably better)

\enrico{General Intro on HBE? Not sure if needed for TPB? Maybe move to discussion?}
Formal models portraying changes in human subsistence have been mostly offered in human behavioral ecology \citep{smith1992,bird2006,kennett2006}, where several mathematical models inspired from evolutionary ecology %we need a better one-line description of HBE. 
 used to predict  based on expected patterns under the assumption of optimality \citep{belovsky1988}. 
While earliest models relied almost exclusively on this assumption, most recent works are increasing discussing how the knowledge of alternative subsistence strategies are acquired %citation needed? Are there such works? Need to check
and whether this process of learning can drive the evolutionary trajectory of a given population \citep{henrich1998}, partly as a synergy with the dual inheritance theory \citep{boyd1985}. 

%Also this bits might be moved to discussion?%
The results of these works often exhibits counter-intuitive outcomes that cannot be predicted by standard assumptions of optimality. For instance, extreme reliance on social learning can promote information parasitism \citep{giraldeau2002}, leading to a decline in the proportion of asocial learners that in certain environmental context can have strongly detrimental effects at the population level \citep{whitehead2009}. Other examples include instances where a reduced effective population size promotes an increase in random drift, and the ultimate loss of adaptive traits \citep{henrich2004}, or instances where neutral traits are jointly transmitted together with adaptive ones \citep{ackland2007}. These examples illustrate how theoretical models based on the dual inheritance theory can extend predictions of the standard optimal foraging theory, integrating dynamics that, at least at a first glance, are regarded as counter-intuitive. 

These include episodes of reversions (e.g. \footnote{citation needed}), where a more recent optimal strategies are abandoned for older suboptimal variants; stable coexistence (e.g. \footnote{citation needed}), where optimal and suboptimal strategies co-exist within a population; and even the complete loss of optimal variants \citep[e.g.][]{henrich2004}. The evolutionary processes leading to these different outcomes are undoubtedly contingent to local historical dynamics, nonetheless we argue that a simple model of social learning in combination with a positive and negative frequency dependent selection can lead all these equilibria.


\enrico{We need to mention that some behaviour might actually enhance the other community (or be detrimental). This can be modelled using a competition coefficient as in \citep{jang2013}. I don't think we need to explore this but it is definitely worth mentioning!}

\enrico{Mutually exclusive trait, but does not necessarily imply HG vs FA mixture of strategy (say 50\%HG vs 50\%F) can be assumed to be a trait. Perhaps this needs to be mentioned earlier when the model is introduced.}

\section{Conclusion}

BlaBlaBla

\section{References}

\bibliographystyle{elsarticle-harv}
\bibliography{references}

\section{Appendix I}

-50,000 iterations
-floating point arithmetic



\end{document}

